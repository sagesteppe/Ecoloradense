% Options for packages loaded elsewhere
\PassOptionsToPackage{unicode}{hyperref}
\PassOptionsToPackage{hyphens}{url}
%
\documentclass[
]{article}
\usepackage{amsmath,amssymb}
\usepackage{lmodern}
\usepackage{iftex}
\ifPDFTeX
  \usepackage[T1]{fontenc}
  \usepackage[utf8]{inputenc}
  \usepackage{textcomp} % provide euro and other symbols
\else % if luatex or xetex
  \usepackage{unicode-math}
  \defaultfontfeatures{Scale=MatchLowercase}
  \defaultfontfeatures[\rmfamily]{Ligatures=TeX,Scale=1}
\fi
% Use upquote if available, for straight quotes in verbatim environments
\IfFileExists{upquote.sty}{\usepackage{upquote}}{}
\IfFileExists{microtype.sty}{% use microtype if available
  \usepackage[]{microtype}
  \UseMicrotypeSet[protrusion]{basicmath} % disable protrusion for tt fonts
}{}
\makeatletter
\@ifundefined{KOMAClassName}{% if non-KOMA class
  \IfFileExists{parskip.sty}{%
    \usepackage{parskip}
  }{% else
    \setlength{\parindent}{0pt}
    \setlength{\parskip}{6pt plus 2pt minus 1pt}}
}{% if KOMA class
  \KOMAoptions{parskip=half}}
\makeatother
\usepackage{xcolor}
\usepackage[margin=1in]{geometry}
\usepackage{graphicx}
\makeatletter
\def\maxwidth{\ifdim\Gin@nat@width>\linewidth\linewidth\else\Gin@nat@width\fi}
\def\maxheight{\ifdim\Gin@nat@height>\textheight\textheight\else\Gin@nat@height\fi}
\makeatother
% Scale images if necessary, so that they will not overflow the page
% margins by default, and it is still possible to overwrite the defaults
% using explicit options in \includegraphics[width, height, ...]{}
\setkeys{Gin}{width=\maxwidth,height=\maxheight,keepaspectratio}
% Set default figure placement to htbp
\makeatletter
\def\fps@figure{htbp}
\makeatother
\setlength{\emergencystretch}{3em} % prevent overfull lines
\providecommand{\tightlist}{%
  \setlength{\itemsep}{0pt}\setlength{\parskip}{0pt}}
\setcounter{secnumdepth}{-\maxdimen} % remove section numbering
\newlength{\cslhangindent}
\setlength{\cslhangindent}{1.5em}
\newlength{\csllabelwidth}
\setlength{\csllabelwidth}{3em}
\newlength{\cslentryspacingunit} % times entry-spacing
\setlength{\cslentryspacingunit}{\parskip}
\newenvironment{CSLReferences}[2] % #1 hanging-ident, #2 entry spacing
 {% don't indent paragraphs
  \setlength{\parindent}{0pt}
  % turn on hanging indent if param 1 is 1
  \ifodd #1
  \let\oldpar\par
  \def\par{\hangindent=\cslhangindent\oldpar}
  \fi
  % set entry spacing
  \setlength{\parskip}{#2\cslentryspacingunit}
 }%
 {}
\usepackage{calc}
\newcommand{\CSLBlock}[1]{#1\hfill\break}
\newcommand{\CSLLeftMargin}[1]{\parbox[t]{\csllabelwidth}{#1}}
\newcommand{\CSLRightInline}[1]{\parbox[t]{\linewidth - \csllabelwidth}{#1}\break}
\newcommand{\CSLIndent}[1]{\hspace{\cslhangindent}#1}
\usepackage{endfloat}
\usepackage{setspace}\doublespacing
\usepackage{lineno}
\linenumbers
\usepackage{booktabs}
\usepackage{longtable}
\usepackage{array}
\usepackage{multirow}
\usepackage{wrapfig}
\usepackage{float}
\usepackage{colortbl}
\usepackage{pdflscape}
\usepackage{tabu}
\usepackage{threeparttable}
\usepackage{threeparttablex}
\usepackage[normalem]{ulem}
\usepackage{makecell}
\usepackage{xcolor}
\ifLuaTeX
  \usepackage{selnolig}  % disable illegal ligatures
\fi
\IfFileExists{bookmark.sty}{\usepackage{bookmark}}{\usepackage{hyperref}}
\IfFileExists{xurl.sty}{\usepackage{xurl}}{} % add URL line breaks if available
\urlstyle{same} % disable monospaced font for URLs
\hypersetup{
  pdftitle={Predicting, habitat suitability, occupancy, and census sizes of a rare plant species using iterative adaptive niche based sampling},
  pdfkeywords={species distribution model, occupancy, census size
estimates,},
  hidelinks,
  pdfcreator={LaTeX via pandoc}}

\title{Predicting, habitat suitability, occupancy, and census sizes of a
rare plant species using iterative adaptive niche based sampling}
\author{Reed Clark Benkendorf\(^1\)\footnote{Author for Correspondence:
  \href{mailto:rbenkendorf@chicagobotanic.org}{\nolinkurl{rbenkendorf@chicagobotanic.org}}},
Jeremie B. Fant\(^1\)\(^,\)\(^2\), Sophie Taddeo\(^3\)\\
\hspace*{0.333em}\(^1\) Chicago Botanic Garden, Glencoe, Illinois 60022,
USA\\
\hspace*{0.333em}\(^2\) Plant Biology and Conservation, Northwestern
University, Evanston, Illinois 60208, USA\\
\hspace*{0.333em}\(^3\) Department of Environmental and Ocean Sciences,
University of San Diego, San Diego, California 92110, USA}
\date{}

\begin{document}
\maketitle
\begin{abstract}
\begin{enumerate}
\def\labelenumi{\arabic{enumi})}
\tightlist
\item
\item
\item
\item
\end{enumerate}
\end{abstract}

\hypertarget{introduction}{%
\section{1 \textbar{} INTRODUCTION}\label{introduction}}

The effects of anthropogenic stressors, e.g.~land use and climate
change, have lead to a global extinction crisis with estimates of the
number of plant species facing extinction ranging from 20-40\% (Brummitt
\emph{et al.} (\protect\hyperlink{ref-brummitt2015green}{2015}), Pimm \&
Joppa (\protect\hyperlink{ref-pimm2015many}{2015}), Nic Lughadha
\emph{et al.} (\protect\hyperlink{ref-nic2020extinction}{2020})).
Determining which plant species to focus our conservation efforts
(e.g.~active restoration, preserve creation, \emph{ex situ} collections)
on requires an array of data which seldom exist for decision makers
(Heywood (\protect\hyperlink{ref-heywood2017plant}{2017})). These data
generally outline simple biological and ecological parameters of species
useful for detailing there rarity and how it's distribution relates to
current and future anthropogenic stressors. Chief amongst these
parameters, are the geographic extent of occurrence (range), the
distribution of suitable habitat - the occupancy of this habitat as well
as the spacing of occupied patches, and the census size of individual
(sub-)populations (Commission
(\protect\hyperlink{ref-natural2001iucn}{2001}), Faber-Langendoen
\emph{et al.} (\protect\hyperlink{ref-faber2012natureserve}{2012}),
\emph{USFWS species status assessment framework}
(\protect\hyperlink{ref-usfws2016ssa}{2016})). While these parameters
are relatively simple, characterizing them can be time consuming,
generally requiring extensive field work and travel to field sites -
hence they oftentimes require proxies or heuristics for estimation in
conservation assessments (Juffe-Bignoli \emph{et al.}
(\protect\hyperlink{ref-juffe2016assessing}{2016}), Bland \emph{et al.}
(\protect\hyperlink{ref-bland2015cost}{2015}), Pelletier \emph{et al.}
(\protect\hyperlink{ref-pelletier2018predicting}{2018})). Environmental
niche models (ENMs or Species Distribution Models SDMs) have made
enormous headway in resolving the former two problems (extent, and
occupancy), however the historic mismatch between the resolution of
variables governing species distributions and the data available to
serve as predictors of environmental conditions have restricted the
interpretation and implementation of these models in highly
heterogeneous environments (taddeo2024grimes, Guisan \emph{et al.}
(\protect\hyperlink{ref-guisan2013predicting}{2013})). Recent advances
in remote-sensing technologies have allowed for the generation of useful
models in these generally biodiverse systems, further these data offer
promise to model additional population parameters (e.g.~census size,
population extents), however the utility and usage of high resolution
have rarely been ground verified or ostensibly reported upon (Chiffard
\emph{et al.} (\protect\hyperlink{ref-chiffard2020anbs}{2020})).

Recently, considerable headway has been made in generating statistically
robust environmental niche models (ENM's) spirited by: recent advances
in collecting high-resolution environmental data, compute power,
digitization of natural history museum records and the acquisition of
citizen science records, and statistical methods especially downsampling
(Markham \emph{et al.}
(\protect\hyperlink{ref-markham2023review}{2023}), Feldman \emph{et al.}
(\protect\hyperlink{ref-feldman2021trends}{2021})).\\
However, ENM's are rarely ground verified, and even more seldom at
landscape scales. Hence, most of our knowledge about producing ENM's
rely on simulated species and data, especially at spatial resolutions
considered relatively coarse to those desired by many analysts. An
historic complication with the implementation and interpretation of
ENM's is a mismatch between the spatial resolution of the independent
variables available to model the species fundamental niche and the
factors governing the true distribution of populations - the realized
niche (Carscadden \emph{et al.}
(\protect\hyperlink{ref-carscadden2020niche}{2020}), Chauvier \emph{et
al.} (\protect\hyperlink{ref-chauvier2022resolution}{2022}), Lembrechts
\emph{et al.}
(\protect\hyperlink{ref-lembrechts2019incorporating}{2019})). Recent
papers have had mixed results regarding the effects of imprecisely
mapped occurrence data on ENM predictions, with indications that models
generated in more heterogenous environments, and at finer resolutions
(e.g.~ca. 3 arc-second relate to 10 arc-minutes (\textasciitilde14.5 km)
at 38*)) suffer minor decreases in model performance (Graham \emph{et
al.} (\protect\hyperlink{ref-graham2008influence}{2008}), Smith \emph{et
al.} (\protect\hyperlink{ref-smith2023including}{2023})) with real
species, while increasing error in mapping has drastic effects on model
predictions with virtually simulated species (Gábor \emph{et al.}
(\protect\hyperlink{ref-gabor2022positional}{2022})).\\
A further mismatch of resolution is the year in which data on geographic
localities were obtained and current conditions which allow for positive
population growth (Bracken \emph{et al.}
(\protect\hyperlink{ref-bracken2022maximizing}{2022})). Historic
occurrence data may now represent conditions which are inhospitable to
the maintenance of populations, or may even represent populations which
even then were simply sinks from more robust populations (Bracken
\emph{et al.} (\protect\hyperlink{ref-bracken2022maximizing}{2022}));
using these records may decrease model performance. Collecting data on
whether areas are favorable to continued recruitment of individuals from
the soil seed bank, are perhaps more astute than whether long-lived
individuals persist.

ENM models generally suffer from having few, generally spatially biased,
occurrence records to serve as dependent variables, which generally fail
to characterize the ecological breadth of the species (Stolar \& Nielsen
(\protect\hyperlink{ref-stolar2015accounting}{2015}), Feeley \& Silman
(\protect\hyperlink{ref-feeley2011keep}{2011})). While many ENM's have
high-performance metrics while tested on small subsets of hold-out data,
they are unlikely to detect many new populations during ground
verification (A. Lee-Yaw \emph{et al.}
(\protect\hyperlink{ref-a2022species}{2022})). To increase the number of
presences, and absences in sites which are relatively similar to those
harboring presences, which can be used for training models iterative
adaptive-niche based sampling (ANBS) has become increasingly employed
(Guisan \emph{et al.} (\protect\hyperlink{ref-guisan2006using}{2006})).
In ANBS cells with high probabilities of occurrence are preferentially
visited, and after each bout of field visits, a new model is fit
incorporating the original data, plus the recently collected data.\\
However, we posit that evaluating the effects of ANBS is complicated as
the species are oftentimes initially `under surveyed'; in these
instances the true distribution of the population is generally poorly
defined - despite it being easy to do so by a naturalist. Here we
showcase the usage of a robust first-stage expert subjective sampling
effort to generate a large pool of random variables for elaboration
after a second modelling bout. Using this process not only allows for
the acquisition of a larger number of presences and absences, but also
allows for verifying coordinate placement, that historic points are
still extent, and for the acquisition of additional data such as census
estimates and life stages (Stockwell \& Peterson
(\protect\hyperlink{ref-stockwell2002effects}{2002}), Wisz \emph{et al.}
(\protect\hyperlink{ref-wisz2008effects}{2008})). For the purposes of
this paper, we believe that it elevates the challenge of future sampling
bouts to be more representative of typical rare species, i.e.~that they
have on occasion been thoroughly surveyed for and hence novel detections
are rare.

An ENM predicts a single outcome; the probability of suitable habitat
for the species (taddeo2024grimes). Although generally the truly desired
insight from an ENM is the species realized distribution.\\
However, the bridge between an ENM and a plant populations presence is
related to the dispersal of propagules and the establishment of the
population, rather than distribution of the fundamental niche alone.\\
To assist practitioners in detecting new populations, or extending the
range of currently known populations, we propose modelling plant
occupancy as a random variable dependent on distance from sites known to
be occupied, the size and shape of the target site, and it's relation to
other suitable habitats. In a simple sense these perspectives link
habitat suitability with the tenets of island biogeography (@).
`Distance' may be defined as euclidean (or Haversine for large
distances), or as a least-cost distance reflecting a generalized surface
which conveys the difficulty for seeds to travel between the nearest
occupied sites and the site of interest (Etherington
(\protect\hyperlink{ref-etherington2016least}{2016})).\\
Landscape metrics postulated to relate to the occupancy at a site
include the patch metrics Core Area Index, roughly reflecting the
probability of a propagule arriving to a larger patch, and the class
metrics proximity and contagion both of which reflect the aggregation
between occupied sites.

Obtaining reliable estimates of plant population census sizes can be
time intensive process because it requires two major pieces of
information 1) measurements of plant density and 2) population extent.
Generally the former is measured using several to many long and narrow
(50x2m) transects, which typically require multiple readers and
recorders, and generate highly spatially auto-correlated data (- ELZINGA
?).\\
In addition to these personnel requirements, many plant species are
endemic to steep, and often loose, slopes, prohibiting the use of these
transects (). Recently several promising field methods for acquiring
density measurements have been developed (Krening \emph{et al.}
(\protect\hyperlink{ref-krening2021sampling}{2021}), Ermakova \emph{et
al.} (\protect\hyperlink{ref-ermakova2021densities}{2021}), Alfaro-Saiz
\emph{et al.} (\protect\hyperlink{ref-alfaro2019optimal}{2019}), Schorr
(\protect\hyperlink{ref-schorr2013using}{2013})), as well as progress in
the application of genomic methods to estimate census sizes
(e.g.~linkage disequilibrium) (CITE). However these methods are focused
on descriptive, rather than predictive processes, where they focus on
estimating the uncertainty in measurements \emph{within} a population,
rather than across the species range. We propose that estimates of plant
density are best generated via methods which allow for the fitting of
statistical models, which can incorporate spatial covariates and predict
estimates and measurements of uncertainty across gridded surfaces
(Oliver \emph{et al.}
(\protect\hyperlink{ref-oliver2012population}{2012}), Doser \emph{et
al.} (\protect\hyperlink{ref-doser2022spabundance}{2024})).

Precisely mapping the boundaries of a population is another time
consuming task, albeit essential for the effective estimates of
population census size. Currently, population boundaries are delineated
by practitioners walking distances (e.g.~1km) in several directions
searching for individuals.\\
However, as the distance from a central location increases the amount of
area to survey increases IN A FASHION\ldots{} Further this approach is
problematic for rare species with many small clusters of inconspicuous
individuals, which require considerable survey effort to chance upon.
High resolution spatial data eventually offer an ability to determine
the extent of individual populations via detection of the edges of
suitable habitat.

\hypertarget{methods}{%
\section{2 \textbar{} METHODS}\label{methods}}

\hypertarget{study-species-system}{%
\subsubsection{Study Species \& System}\label{study-species-system}}

\emph{Eriogonum coloradense} Small (Polygonaceae) is a synoecious mat
forming perennial herb endemic to the Central Southern Rocky Mountains
in Colorado, U.S.A. It's known geographic range covers XX
km\textsuperscript{2}, has 26 formally described populations, and is
thought to occur across a range of elevation, slopes, aspects, soil
types and habitats. The elevation range from which it is xx - xx, and
the broad habitat types it's known from include: high elevation
sagebrush steppe, sub-alpine grasslands, and alpine slopes. It is
treated as an S3/G3 species by NatureServe, and a Tier 2 species by
State Wildlife Action Plan by the Colorado Parks and Wildlife Service.

\hypertarget{data-acqusition}{%
\subsubsection{Data Acqusition}\label{data-acqusition}}

Dependent data were gathered from iNaturalist (80 records) and the
Consortium of Southern Rockies herbaria (131 records) ({`Southern rocky
mountain herbaria portal'}, INATURALIST). These data were manually
reviewed and 16 herbarium records which had low geolocation quality, or
which were georeferenced to localities which did not match their
herbarium labels were removed.

Digital Elevation Models at 3arc (type), and 1arc (type), and Digital
Elevation Products at 1/3 arc, and 1m, resolution were acquired from the
United States Geological Survey and clipped to the domain of analysis (a
rectangle buffered 16km??? (10 mi). beyond any known population). A 3m
resolution digital elevation model, which is not available at a native
resolution from the USGS for this area, was created by bilinear
resampling of the 1m and 10m data. 1m resolution data were only
available for roughly XX\% of the domain (which contained \%\% of known
populations), the remainder of the species domains 3m data was created
by linear resampling of the 10m DEP, a comparison of these results in
the overlapping region are present in appendix xx. These elevation
products were used to create all geomorphology data sets using
whiteboxTools (Wu \& Brown
(\protect\hyperlink{ref-wu2022whitebox}{2022})). Vegetation cover data
were made by combining the raster data into continuous covers: Forested,
Shrub, and Herbaceous vegetation Tuanmu \& Jetz
(\protect\hyperlink{ref-tuanmu2014global}{2014}).

ClimateNA was used to create a data set at 3 arc-second resolution which
then underwent simple bilinear interpolation to generate products at the
finer resolutions (Wang \emph{et al.}
(\protect\hyperlink{ref-wang2016locally}{2016})). Gray co-occurrence
level matrices were produced using the glcm r package using 2023 NAIP
aerial imagery, which underwent bilinear resampling to each resolution,
using default setting but with windows of 5 in both directions (Zvoleff
(\protect\hyperlink{ref-zvoleff2020glcm}{2020})).

\hypertarget{ground-verification}{%
\subsubsection{Ground Verification}\label{ground-verification}}

The first round of ground verification was carried out from
June-September 2024. All pre-existing occurrence points were considered
candidates for revisits and all X trails leading to them were marked as
SAMPLE UNITS. Each trail was manually mapped, and buffered 45m in each
direction and XXX random points were drawn, thinned to distances
\textgreater{} XXXm, leaving XXX random plots for assessment. XX trails
were visited, allowing for the assessment of XX random points and XX
occurrences. When conducting field work, all presences of \emph{E.
coloradense} were opportunistically noted, and to better describe the
spread of the population points were subjectively placed ca. 30-50m from
the previous one until passing out of the population (n = ).
Additionally subjectively placed absences were also collected in areas
which seemed favorable, or were in close proximity, to \emph{E.
coloradense} individuals; this occurred both in field (n = ) and through
use of aerial imagery on a computer afterwards (n = ).

Adaptive Niche-Based Sampling was carried out in July of 2025. To
determine whether adaptive sampling performed better than alternative
sampling regimes (e.g.~stratified, random), and if it could be improved
by occupancy modelling 30 points were selected for each of the
aforementioned sampling schema plus 30 occupancy points.

\hypertarget{comparision-of-different-spatial-resolutions}{%
\subsubsection{Comparision of Different Spatial
Resolutions}\label{comparision-of-different-spatial-resolutions}}

Environmental Niche Models were generated at five different spatial
resolutions, 3 (\textasciitilde72m), 1 (\textasciitilde24m), and 1/3
(\textasciitilde8m) arc-seconds, and 3 and 1m resolution. The details of
modelling were similar for each resolution.

Records were thinned to the distance of an hypotenuse of a cell to avoid
replicates (Aiello-Lammens \emph{et al.}
(\protect\hyperlink{ref-aeillo2015spthin}{2015})). XXX Absence records
were generated using the background function with method environmental
distance from sdm (Naimi \& Araujo
(\protect\hyperlink{ref-naimi2016sdm}{2016})), these records were then
manually reviewed and six records which were deemed in areas which may
be possible presences were removed, after this the records were randomly
sampled to reduce the data set size to XX records.\\
After the first iteration of modelling all additional presences and
absences were thinned via a similar manner and combined with the
original absence records. `Presences' which had greater coordinate
uncertainty than the resolution of modelling were removed.

compare the results of each spatial resolution at each of the 3 stages
(naive, expert, adaptive)

\hypertarget{adaptive-and-occupancy-based-field-sampling}{%
\subsubsection{Adaptive and Occupancy Based Field
Sampling}\label{adaptive-and-occupancy-based-field-sampling}}

500 stratified points ranging from 1-100\% probability of suitable
habitat were generated using sample (terra). Occupancy scores for each
of these points were then calculated using (whiteboxtools). 200 points
which maximized the spread of values along both dimensions (habitat,
occupancy) were then chosen for ground verification.

\hypertarget{plant-density}{%
\subsubsection{Plant Density}\label{plant-density}}

Plant density was modelled using five methods, a generalized linear
model (GLM) with a poisson error distribution and spatial
autocorrelation structures, spAbundance, Random Forest and XgBoost
regression.

\hypertarget{species-occupancy}{%
\subsubsection{Species Occupancy}\label{species-occupancy}}

\hypertarget{comparision-of-juvenile-and-mature-plant-models}{%
\subsection{Comparision of Juvenile and Mature Plant
Models}\label{comparision-of-juvenile-and-mature-plant-models}}

Using the top performing model resolution () models were refit using
only either juvenile or mature plant presences, the number of mature
presences were limited to the number of juvenile occurrences. Models
were fit at 6 sample sizes (n = 15, 30, 50, 75, 125, 200) 15 times each
using a randomly sampled 60\%-40\% train-test split of data.

\hypertarget{simulations-of-sample-size}{%
\subsection{Simulations of Sample
Size}\label{simulations-of-sample-size}}

The effect of sample size on model performance was simulated at 8 sample
sizes (n = 15, 30, 50, 75, 125, 200, 300, 400) each 25 times, using a
randomly sampled 60\%-40\% train-test split of data.

\hypertarget{simulations-of-coordinate-errors}{%
\subsection{Simulations of Coordinate
Errors}\label{simulations-of-coordinate-errors}}

The effect of sample size on model performance was simulated at 6 sample
sizes (n = 15, 30, 50, 75, 125, 200), with three proportions of records
in error (0\%, 5\%, 10\%, 20\%), and 3 levels of coordinate uncertainty
(0m, 10m, 100m, 1000m), at each of the 5 resolutions 25 times.

\hypertarget{results}{%
\subsection{3 \textbar{} RESULTS}\label{results}}

\hypertarget{comparision-of-different-spatial-resolutions-1}{%
\subsubsection{Comparision of Different Spatial
Resolutions}\label{comparision-of-different-spatial-resolutions-1}}

\hypertarget{ground-verification-1}{%
\subsubsection{Ground Verification}\label{ground-verification-1}}

\hypertarget{plant-density-1}{%
\subsubsection{Plant Density}\label{plant-density-1}}

\hypertarget{species-occupancy-1}{%
\subsubsection{Species Occupancy}\label{species-occupancy-1}}

\hypertarget{comparision-of-juvenile-and-mature-plant-models-1}{%
\subsection{Comparision of Juvenile and Mature Plant
Models}\label{comparision-of-juvenile-and-mature-plant-models-1}}

\hypertarget{simulations-of-sample-size-1}{%
\subsection{Simulations of Sample
Size}\label{simulations-of-sample-size-1}}

\hypertarget{simulations-of-coordinate-errors-1}{%
\subsection{Simulations of Coordinate
Errors}\label{simulations-of-coordinate-errors-1}}

\hypertarget{discussion}{%
\subsection{4 \textbar{} DISCUSSION}\label{discussion}}

\hypertarget{conclusions}{%
\subsection{5 \textbar{} CONCLUSIONS}\label{conclusions}}

\hypertarget{acknowledgments}{%
\subsection{6 \textbar{} ACKNOWLEDGMENTS}\label{acknowledgments}}

\hypertarget{references}{%
\subsection*{REFERENCES}\label{references}}
\addcontentsline{toc}{subsection}{REFERENCES}

\hypertarget{refs}{}
\begin{CSLReferences}{1}{0}
\leavevmode\vadjust pre{\hypertarget{ref-a2022species}{}}%
A. Lee-Yaw, J., L. McCune, J., Pironon, S. \& N. Sheth, S. (2022).
Species distribution models rarely predict the biology of real
populations. \emph{Ecography}, \textbf{2022}, e05877.

\leavevmode\vadjust pre{\hypertarget{ref-aeillo2015spthin}{}}%
Aiello-Lammens, M.E., Boria, R.A., Radosavljevic, A. \& Vilela, B.
(2015).
\href{https://onlinelibrary.wiley.com/doi/10.1111/ecog.01132}{{spThin}:
An {R} package for spatial thinning of species occurrence records for
use in ecological niche models}. \emph{Ecography}, \textbf{38},
541--545.

\leavevmode\vadjust pre{\hypertarget{ref-alfaro2019optimal}{}}%
Alfaro-Saiz, E., Granda, V., Rodrı́guez, A., Alonso-Redondo, R. \&
Garcı́a-González, M.E. (2019). Optimal census method to estimate
population sizes of species growing on rock walls: The case of mature
primula pedemontana. \emph{Global Ecology and Conservation},
\textbf{17}, e00563.

\leavevmode\vadjust pre{\hypertarget{ref-bland2015cost}{}}%
Bland, L.M., Orme, C.D.L., Bielby, J., Collen, B., Nicholson, E. \&
McCarthy, M.A. (2015). Cost-effective assessment of extinction risk with
limited information. \emph{Journal of Applied Ecology}, \textbf{52},
861--870.

\leavevmode\vadjust pre{\hypertarget{ref-bracken2022maximizing}{}}%
Bracken, J.T., Davis, A.Y., O'Donnell, K.M., Barichivich, W.J., Walls,
S.C. \& Jezkova, T. (2022). Maximizing species distribution model
performance when using historical occurrences and variables of varying
persistency. \emph{Ecosphere}, \textbf{13}, e3951.

\leavevmode\vadjust pre{\hypertarget{ref-brummitt2015green}{}}%
Brummitt, N.A., Bachman, S.P., Griffiths-Lee, J., Lutz, M., Moat, J.F.,
Farjon, A., Donaldson, J.S., Hilton-Taylor, C., Meagher, T.R.,
Albuquerque, S. \& others. (2015). Green plants in the red: A baseline
global assessment for the IUCN sampled red list index for plants.
\emph{PloS one}, \textbf{10}, e0135152.

\leavevmode\vadjust pre{\hypertarget{ref-carscadden2020niche}{}}%
Carscadden, K.A., Emery, N.C., Arnillas, C.A., Cadotte, M.W., Afkhami,
M.E., Gravel, D., Livingstone, S.W. \& Wiens, J.J. (2020). Niche
breadth: Causes and consequences for ecology, evolution, and
conservation. \emph{The Quarterly Review of Biology}, \textbf{95},
179--214.

\leavevmode\vadjust pre{\hypertarget{ref-chauvier2022resolution}{}}%
Chauvier, Y., Descombes, P., Guéguen, M., Boulangeat, L., Thuiller, W.
\& Zimmermann, N.E. (2022). Resolution in species distribution models
shapes spatial patterns of plant multifaceted diversity.
\emph{Ecography}, \textbf{2022}, e05973.

\leavevmode\vadjust pre{\hypertarget{ref-chiffard2020anbs}{}}%
Chiffard, J., Marciau, C., Yoccoz, N.G., Mouillot, F., Duchateau, S.,
Nadeau, I., Fontanilles, P. \& Besnard, A. (2020).
\href{https://doi.org/10.1111/2041-210X.13399}{Adaptive niche-based
sampling to improve ability to find rare and elusive species:
Simulations and field tests}. \emph{Methods in Ecology and Evolution},
\textbf{11}, 899--909.

\leavevmode\vadjust pre{\hypertarget{ref-natural2001iucn}{}}%
Commission, N.Resources.S.S. (2001). \emph{IUCN red list categories and
criteria}. IUCN.

\leavevmode\vadjust pre{\hypertarget{ref-doser2022spabundance}{}}%
Doser, J.W., Finley, A.O., Kéry, M. \& Zipkin, E.F. (2024).
\href{https://doi.org/10.1111/2041-210X.14332}{spAbundance: An r package
for single-species and multi-species spatially explicit abundance
models}. \emph{Methods in Ecology and Evolution}, \textbf{15},
1024--1033.

\leavevmode\vadjust pre{\hypertarget{ref-ermakova2021densities}{}}%
Ermakova, A., Trout, K., Terry, M.K., Whiting, C.V., Clubbe, C. \&
Fowler, N. (2021). Densities, plant sizes, and spatial distributions of
six wild populations of lophophora williamsii (cactaceae) in texas, USA.
\emph{Journal of the Botanical Research Institute of Texas},
\textbf{15}, 149--160.

\leavevmode\vadjust pre{\hypertarget{ref-etherington2016least}{}}%
Etherington, T.R. (2016). Least-cost modelling and landscape ecology:
Concepts, applications, and opportunities. \emph{Current Landscape
Ecology Reports}, \textbf{1}, 40--53.

\leavevmode\vadjust pre{\hypertarget{ref-faber2012natureserve}{}}%
Faber-Langendoen, D., Nichols, J., Master, L., Snow, K., Tomaino, A.,
Bittman, R., Hammerson, G., Heidel, B., Ramsay, L., Teucher, A. \&
others. (2012). NatureServe conservation status assessments: Methodology
for assigning ranks. \emph{NatureServe, Arlington, VA}.

\leavevmode\vadjust pre{\hypertarget{ref-feeley2011keep}{}}%
Feeley, K.J. \& Silman, M.R. (2011). Keep collecting: Accurate species
distribution modelling requires more collections than previously
thought. \emph{Diversity and distributions}, \textbf{17}, 1132--1140.

\leavevmode\vadjust pre{\hypertarget{ref-feldman2021trends}{}}%
Feldman, M.J., Imbeau, L., Marchand, P., Mazerolle, M.J., Darveau, M. \&
Fenton, N.J. (2021). Trends and gaps in the use of citizen science
derived data as input for species distribution models: A quantitative
review. \emph{PloS one}, \textbf{16}, e0234587.

\leavevmode\vadjust pre{\hypertarget{ref-gabor2022positional}{}}%
Gábor, L., Jetz, W., Lu, M., Rocchini, D., Cord, A., Malavasi, M.,
Zarzo-Arias, A., Barták, V. \& Moudrỳ, V. (2022). Positional errors in
species distribution modelling are not overcome by the coarser grains of
analysis. \emph{Methods in Ecology and Evolution}, \textbf{13},
2289--2302.

\leavevmode\vadjust pre{\hypertarget{ref-graham2008influence}{}}%
Graham, C.H., Elith, J., Hijmans, R.J., Guisan, A., Townsend Peterson,
A., Loiselle, B.A. \& Group, N.P.S.D.W. (2008). The influence of spatial
errors in species occurrence data used in distribution models.
\emph{Journal of Applied Ecology}, \textbf{45}, 239--247.

\leavevmode\vadjust pre{\hypertarget{ref-guisan2006using}{}}%
Guisan, A., Broennimann, O., Engler, R., Vust, M., Yoccoz, N.G.,
Lehmann, A. \& Zimmermann, N.E. (2006). Using niche-based models to
improve the sampling of rare species. \emph{Conservation biology},
\textbf{20}, 501--511.

\leavevmode\vadjust pre{\hypertarget{ref-guisan2013predicting}{}}%
Guisan, A., Tingley, R., Baumgartner, J.B., Naujokaitis-Lewis, I.,
Sutcliffe, P.R., Tulloch, A.I., Regan, T.J., Brotons, L.,
McDonald-Madden, E., Mantyka-Pringle, C. \& others. (2013). Predicting
species distributions for conservation decisions. \emph{Ecology
letters}, \textbf{16}, 1424--1435.

\leavevmode\vadjust pre{\hypertarget{ref-heywood2017plant}{}}%
Heywood, V.H. (2017). Plant conservation in the anthropocene--challenges
and future prospects. \emph{Plant diversity}, \textbf{39}, 314--330.

\leavevmode\vadjust pre{\hypertarget{ref-juffe2016assessing}{}}%
Juffe-Bignoli, D., Brooks, T.M., Butchart, S.H., Jenkins, R.B., Boe, K.,
Hoffmann, M., Angulo, A., Bachman, S., Böhm, M., Brummitt, N. \& others.
(2016). Assessing the cost of global biodiversity and conservation
knowledge. \emph{PLoS One}, \textbf{11}, e0160640.

\leavevmode\vadjust pre{\hypertarget{ref-krening2021sampling}{}}%
Krening, P.P., Dawson, C.A., Holsinger, K.W. \& Willoughby, J.W. (2021).
A sampling-based approach to estimating the minimum population size of
the federally threatened colorado hookless cactus (sclerocactus
glaucus). \emph{Natural Areas Journal}, \textbf{41}, 4--10.

\leavevmode\vadjust pre{\hypertarget{ref-lembrechts2019incorporating}{}}%
Lembrechts, J.J., Nijs, I. \& Lenoir, J. (2019). Incorporating
microclimate into species distribution models. \emph{Ecography},
\textbf{42}, 1267--1279.

\leavevmode\vadjust pre{\hypertarget{ref-markham2023review}{}}%
Markham, K., Frazier, A.E., Singh, K.K. \& Madden, M. (2023). A review
of methods for scaling remotely sensed data for spatial pattern
analysis. \emph{Landscape Ecology}, \textbf{38}, 619--635.

\leavevmode\vadjust pre{\hypertarget{ref-naimi2016sdm}{}}%
Naimi, B. \& Araujo, M.B. (2016).
\href{https://doi.org/10.1111/ecog.01881}{Sdm: A reproducible and
extensible r platform for species distribution modelling}.
\emph{Ecography}, \textbf{39}, 368--375.

\leavevmode\vadjust pre{\hypertarget{ref-nic2020extinction}{}}%
Nic Lughadha, E., Bachman, S.P., Leão, T.C., Forest, F., Halley, J.M.,
Moat, J., Acedo, C., Bacon, K.L., Brewer, R.F., Gâteblé, G. \& others.
(2020). Extinction risk and threats to plants and fungi. \emph{Plants,
People, Planet}, \textbf{2}, 389--408.

\leavevmode\vadjust pre{\hypertarget{ref-oliver2012population}{}}%
Oliver, T.H., Gillings, S., Girardello, M., Rapacciuolo, G., Brereton,
T.M., Siriwardena, G.M., Roy, D.B., Pywell, R. \& Fuller, R.J. (2012).
Population density but not stability can be predicted from species
distribution models. \emph{Journal of Applied Ecology}, \textbf{49},
581--590.

\leavevmode\vadjust pre{\hypertarget{ref-pelletier2018predicting}{}}%
Pelletier, T.A., Carstens, B.C., Tank, D.C., Sullivan, J. \& Espı́ndola,
A. (2018). Predicting plant conservation priorities on a global scale.
\emph{Proceedings of the National Academy of Sciences}, \textbf{115},
13027--13032.

\leavevmode\vadjust pre{\hypertarget{ref-pimm2015many}{}}%
Pimm, S.L. \& Joppa, L.N. (2015). How many plant species are there,
where are they, and at what rate are they going extinct? \emph{Annals of
the Missouri Botanical Garden}, \textbf{100}, 170--176.

\leavevmode\vadjust pre{\hypertarget{ref-schorr2013using}{}}%
Schorr, R.A. (2013). Using distance sampling to estimate density and
abundance of saussurea weberi hult{é}n (weber's saw-wort). \emph{The
Southwestern Naturalist}, \textbf{58}, 378--383.

\leavevmode\vadjust pre{\hypertarget{ref-smith2023including}{}}%
Smith, A.B., Murphy, S.J., Henderson, D. \& Erickson, K.D. (2023).
Including imprecisely georeferenced specimens improves accuracy of
species distribution models and estimates of niche breadth. \emph{Global
Ecology and Biogeography}, \textbf{32}, 342--355.

\leavevmode\vadjust pre{\hypertarget{ref-soro2024}{}}%
Southern rocky mountain herbaria portal.

\leavevmode\vadjust pre{\hypertarget{ref-stockwell2002effects}{}}%
Stockwell, D.R. \& Peterson, A.T. (2002). Effects of sample size on
accuracy of species distribution models. \emph{Ecological modelling},
\textbf{148}, 1--13.

\leavevmode\vadjust pre{\hypertarget{ref-stolar2015accounting}{}}%
Stolar, J. \& Nielsen, S.E. (2015). Accounting for spatially biased
sampling effort in presence-only species distribution modelling.
\emph{Diversity and Distributions}, \textbf{21}, 595--608.

\leavevmode\vadjust pre{\hypertarget{ref-tuanmu2014global}{}}%
Tuanmu, M.-N. \& Jetz, W. (2014). A global 1-km consensus land-cover
product for biodiversity and ecosystem modelling. \emph{Global Ecology
and Biogeography}, \textbf{23}, 1031--1045.

\leavevmode\vadjust pre{\hypertarget{ref-usfws2016ssa}{}}%
\emph{USFWS species status assessment framework: An integrated
analytical framework for conservation 3.4}. (2016). U.S. Fish; Wildlife
Service, 5275 VA-7, Falls Church, VA 22041.

\leavevmode\vadjust pre{\hypertarget{ref-wang2016locally}{}}%
Wang, T., Hamann, A., Spittlehouse, D. \& Carroll, C. (2016). Locally
downscaled and spatially customizable climate data for historical and
future periods for north america. \emph{PloS one}, \textbf{11},
e0156720.

\leavevmode\vadjust pre{\hypertarget{ref-wisz2008effects}{}}%
Wisz, M.S., Hijmans, R., Li, J., Peterson, A.T., Graham, C., Guisan, A.
\& Group, N.P.S.D.W. (2008). Effects of sample size on the performance
of species distribution models. \emph{Diversity and distributions},
\textbf{14}, 763--773.

\leavevmode\vadjust pre{\hypertarget{ref-wu2022whitebox}{}}%
Wu, Q. \& Brown, A. (2022).
\emph{\href{https://CRAN.R-project.org/package=whitebox}{'Whitebox':
'WhiteboxTools' r frontend}}.

\leavevmode\vadjust pre{\hypertarget{ref-zvoleff2020glcm}{}}%
Zvoleff, A. (2020).
\emph{\href{https://CRAN.R-project.org/package=glcm}{Glcm: Calculate
textures from grey-level co-occurrence matrices (GLCMs)}}.

\end{CSLReferences}

\end{document}
